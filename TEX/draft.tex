\documentclass{report}
\usepackage[utf8]{inputenc}

\title{Report on Certificate Tools}
\author{Oliver Bodnár}
\date{August 2016}

\begin{document}

\maketitle

\tableofcontents
\newpage
\section{Overview}

\section{OpenSSL}
\subsection{type}
OpenSSL uses PKCS12 to store keys and or certificates. Certificates and keys  made in OpenSSL however are being made into PEM or DER encoded file. Said keys/certificates can then be stored inside single .pfx file made with 

\subsection{Viewing stored certificates}
PEM encoded certificates (.pem|.cer|.crt):
        openssl x509 -in sample_cert.extention -text -noout

DER encoded certificates (.der):
        openssl x509 -in certificate.der -inform der -text -noout

Importing PEM or DER encoded keys or certificates into PKCS#12 file:

        openssl pkcs12 -export -in file.pem -out file.p12 -name "My Certificate" -certfile othercerts.pem

        -certfile othercerts.pem option is used only if importing more certificates into 1 PKCS#12 file is wanted.

\subsection{License}
OpenSSL is free to use commercialy, however creating CA is not advised - OPENSSL O'REILY page 59 - but also hinted in manual pages - The ca command is quirky and at times downright unfriendly.

The ca utility was originally meant as an example of how to do things in a CA . It was not supposed to be used as a full blown CA itself: nevertheless some people are using it for this purpose.

The ca command is effectively a single user command: no locking is done on the various files and attempts to run more than one ca command on the same database can have unpredictable results. 

\subsection{Generation of / signing a certificate}

\subsubsection{Generate keys and self-signed certificate}
It is possible to generate private key and self signed certificate with command openssl req -x509 -sha256 -nodes -days 365 -newkey rsa:2048 -keyout privateKey.key -out certificate_name.crt

\subsubsection{Generate keys and Certificate Signing Request}
It is possible to generate private key and CSR with command openssl req -out CSR.csr -new -newkey rsa:2048 -nodes -keyout privateKey.key

\subsubsection{Certificate Signing Request signing}
CSR signing can be done by CA created in openSSL by command
openssl ca -config ca/openssl.cnf \
      -extensions server_cert -days 375 -notext -md sha256 \
      -in ca/csr/www.example.com.csr.pem \
      -out ca/certs/www.example.com.cert.pem

\subsubsection{Create combination of private key and signed chain}
First creation of the request is needed, after that request must be signed. To combine them together the creation of pkcs#12 file is needed with commands:
        openssl pkcs12 -export -out outfilename.p12 -in signed_certificate.crt -inkey privateKey.key -chain -CAfile ca-all.crt -password pass:PASSWORD

\subsection{Specification of key length and algorithm}
Key length and algorithm is specified while generating key, currently OpenSSL supports Public-key cryptography algorithms:
RSA, DSA, Diffie-Hellman key exchange, Elliptic Curve

In the past also support for GOST R 34.10-2001 but as of January 2016 deprecated (https://mta.openssl.org/pipermail/openssl-commits/2016-January/003023.html )

\subsection{Specification of validity}
Validity is specified by CA at the moment of signing CSR.

\subsection{Basic Constraints}
Specification of constraints is made by changing basicConstraints part in [ v3_req ] part of openssl.cnf of CA
[ v3_req ]

basicConstraints=critical,CA:BOOL_VAL, pathlen:<maxChainLengthInteger>

\subsection{Seting Subject Alternative Name for end certificates}
To set SAN the change of subjectAltName in openssl.cnf is needed

[ v3_req ]
subjectAltName = @alt_names

[alt_names]
DNS.1 = example1.com
DNS.2 = example2.com
DNS.<next_number> = dns_webaddress.com

\subsection{Using Cryptographic Service Provider}
//TO DO

\subsection{Conversions}
PKCS12 and JKS conversion in OpenSSL
//AFAIK cannot be done using only OpenSSL needed to be tested/researched more

\subsection{Export certificate only from PKCS12 file}
It can be done with command:
openssl pkcs12 -in [yourfile.pfx] -clcerts -nokeys -out [certificate.crt] 

\subsection{Export private key only from PKCS12 file}
To export private key use command:
openssl pkcs12 -nocerts -in [filename.pfx] -out [site.key]

\subsection{Importing certificates and private keys into PKCS12 file}
//should be possible with openssl pkcs12 -export -in file.pem -out file.p12 -name "My Certificate" -certfile othercerts.pem but need to test it.




\section{Keygen}

\end{document}
