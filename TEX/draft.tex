\documentclass{report}
\usepackage[utf8]{inputenc}

\title{Report on Certificate Tools}
\author{Oliver Bodnár}
\date{August 2016}

\begin{document}

\maketitle

\tableofcontents
\newpage
\part{Overview}
\chapter{Tools Overview Table}


\part{Tools}
\chapter{OpenSSL}
\section{type}
OpenSSL uses PKCS12 to store keys and or certificates. Certificates and keys  made in OpenSSL however are being made into PEM or DER encoded file. Said keys/certificates can then be stored inside single .pfx file made with 

\section{Viewing stored certificates}
PEM encoded certificates (.pem|.cer|.crt):
\begin{verbatim}
        openssl x509 -in sample_cert.extention -text -noout
\end{verbatim}
DER encoded certificates (.der):
\begin{verbatim}
        openssl x509 -in certificate.der -inform der -text -noout
\end{verbatim}
Importing PEM or DER encoded keys or certificates into PKCS#12 file:
\begin{verbatim}
        openssl pkcs12 -export -in file.pem -out file.p12 -name "My Certificate" -certfile othercerts.pem
\end{verbatim}
        certfile option is used only if importing more certificates into a single PKCS#12 file is wanted.

\section{License}
OpenSSL is free to use commercialy, however creating CA is not advised - OPENSSL O'REILY page 59 - but also hinted in manual pages.
The ca command is quirky and at times downright unfriendly.

The ca utility was originally meant as an example of how to do things in a CA . It was not supposed to be used as a full blown CA itself: nevertheless some people are using it for this purpose.

The ca command is effectively a single user command: no locking is done on the various files and attempts to run more than one ca command on the same database can have unpredictable results. 

\section{Generation of / signing a certificate}

\subsection{Generate keys and self-signed certificate}
It is possible to generate private key and self signed certificate with 
\begin{verbatim}
openssl req -x509 -sha256 -nodes -days 365 -newkey rsa:2048 -keyout privateKey.key -out certificate_name.crt
\end{verbatim}

\subsection{Generate keys and Certificate Signing Request}
It is possible to generate private key and CSR with
\begin{verbatim}
openssl req -out CSR.csr -new -newkey rsa:2048 -nodes -keyout privateKey.key
\end{verbatim}

\subsection{Certificate Signing Request signing}
CSR signing can be done by CA created in openSSL by command
\begin{verbatim}
openssl ca -config ca/openssl.cnf -extensions server_cert -days 375 -notext -md sha256 -in ca/csr/www.example.com.csr.pem -out ca/certs/www.example.com.cert.pem
\end{verbatim}

\subsection{Create combination of private key and signed chain}
First creation of the request is needed, after that request must be signed. To combine them together the creation of pkcs#12 file is needed with commands:
\begin{verbatim}
        openssl pkcs12 -export -out outfilename.p12 -in signed_certificate.crt -inkey privateKey.key -chain -CAfile ca-all.crt -password pass:PASSWORD
\end{verbatim}

\subsection{Specification of key length and algorithm}
Key length and algorithm is specified while generating key, currently OpenSSL supports Public-key cryptography algorithms:
RSA, DSA, Diffie-Hellman key exchange, Elliptic Curve

In the past also support for GOST R 34.10-2001 but as of January 2016 deprecated (https://mta.openssl.org/pipermail/openssl-commits/2016-January/003023.html )

\subsection{Specification of validity}
Validity is specified by CA at the moment of signing CSR.

\subsection{Basic Constraints}
Specification of constraints is made by changing basicConstraints part in [ v3_req ] part of openssl.cnf of CA
\begin{verbatim}
[ v3_req ]

basicConstraints=critical,CA:BOOL_VAL,\
 pathlen:<maxChainLengthInteger>
\end{verbatim}

\subsection{Seting Subject Alternative Name for end certificates}
To set SAN the change of subjectAltName in openssl.cnf is needed
\begin{verbatim}
[ v3_req ]
subjectAltName = @alt_names

[alt_names]
DNS.1 = example1.com
DNS.2 = example2.com
DNS.<next_number> = dns_webaddress.com
\end{verbatim}

\subsection{Using Cryptographic Service Provider}
TO DO

\section{Conversions}
\subsection{PKCS12 and JKS conversion in OpenSSL}




\subsection{Export certificate only from PKCS12 file}
It can be done with command:
\begin{verbatim}
openssl pkcs12 -in yourfile.pfx -clcerts \
 -nokeys -out certificate.crt 
\end{verbatim}

\subsection{Export private key only from PKCS12 file}
To export private key use command:
\begin{verbatim}
openssl pkcs12 -nocerts -in filename.pfx -out site.key
\end{verbatim}

\subsection{Importing into PKCS12 file}
should be possible with 
\begin{verbatim}
openssl pkcs12 -export -in file.pem -out file.p12 \

-name "My Certificate" -certfile \
othercerts.pem but need to test it.
\end{verbatim}



\chapter{Keygen}

\end{document}
